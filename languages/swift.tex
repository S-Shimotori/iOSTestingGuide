\lstdefinelanguage{swift}{
    sensitive=true,
    keywords=[1]{associatedtype,class,deinit,enum,extension,fileprivate,func,import,init,inout,internal,let,open,operator,private,protocol,public,static,struct,subscript,typealias,var,break,case,continue,default,defer,do,else,fallthrough,for,guard,if,in,repeat,return,switch,where,while,as,Any,catch,false,is,nil,rethrows,super,self,Self,throw,throws,true,try,\_,associativity,convenience,dynamic,didSet,final,get,infix,indirect,lazy,left,mutating,none,nonmutating,optional,override,postfix,precedence,prefix,Protocol,required,right,set,Type,unowned,weak,willSet},
    keywords=[2]{\#available,\#colorLiteral,\#column,\#else,\#elseif,\#endif,\#file,\#fileLiteral,\#function,\#if,\#imageLiteral,\#line,\#selector,\#sourceLocation},
    numbers=left,
    numberstyle=\scriptsize,
    stepnumber=1,
    numbersep=8pt,
    showstringspaces=false,
    breaklines=true,
    morecomment=[l]{//},
    morecomment=[s]{/*}{*/},
    keywordstyle=[1]\color{konpeki},
    keywordstyle=[2]\color{rindo},
    commentstyle=\color{usumoegi},
    stringstyle=\color{enji},
    morestring=[b]",%"
}

\renewcommand{\lstlistingname}{プログラム}
\lstset{language=swift,
    basicstyle=\ttfamily\scriptsize,
    commentstyle=\textit,
    classoffset=1,
    keywordstyle=\bfseries,
    frame=tRBl,
    framesep=5pt,
    showstringspaces=false,
    numbers=left,
    stepnumber=1,
    numberstyle=\tiny,
    tabsize=2
}

