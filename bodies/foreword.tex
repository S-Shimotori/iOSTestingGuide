勉強会や懇親会では,しばしば「テストを書いていますか?」という会話が行われます.その回答に耳を傾けると,

\begin{screen}%
\noindent 「書いてないんですよね」 \\
「モデルのテストなら書いてます」 \\
「JUnitなら授業で習ったけど」 \\
「ユニットテストだけ書いてれば十分じゃないかな」 \\
「書いたことないです」
\end{screen}

\noindent というネガティブなものが目立ちます.上2つは社会人の方の,下3つは学生の方の実際の回答です.特に学生や新社会人をはじめとする初心者の方に憂慮を覚えるところでしょう.なぜならテストを学ぶ機会とモチベーションは限られているからです.筆者の大学では,テストを含む開発手法の授業は必修ではありません.必修の学校があったとしても,プログラミングパラダイムやアルゴリズム,データ構造を学んで余った時間で行われることでしょう.また,インターネット上には新卒に書籍を勧めるブログ記事が多々ありますが,そのリストの中にテストに関する書籍がないものもしばしば見受けられます.テストの有用性を学び,実装して理解する環境は,非常に貴重なのではないでしょうか.

そこで,本書の目的を以下の2点に定めます.

\begin{enumerate}
    \item カンファレンス発表駆動学習\footnote{発表資料を作るべく自分を追い込む学習法のこと(?)}により,筆者がテスト手法に関して知識を得る
    \item 読者がテストについて知識を深める機会を用意する
\end{enumerate}

\noindent 本書により,テストに関するセッションの後,懇親会で「テストやってます!」という会話が増えることを願います.

(2017年10月2日追記)@tarappoさんのご好意で\href{https://orecon.connpass.com/event/63769/}{俺コン Vol.1 / Day. 1 - connpass}にて本書の一部を発表しました.ありがとうございます.

