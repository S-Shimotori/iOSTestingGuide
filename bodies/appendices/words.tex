本付録ではISO/IEC/IEEE 29119\cite{ieee29119-1}\cite{ieee29119-2}\cite{ieee29119-3}\cite{ieee29119-4}\cite{ieee29119-5}で定められるソフトウェアテストの用語を紹介する.用語の日本語訳はJSTQB®技術委員会の翻訳\cite{jstqb_words}に従う.

\begin{description}
    \item[accessibility testing(アクセシビリティテスト)\index{accessibility testing}]usability testingの一種.障害者を含む多くのユーザがtest itemを使えるかどうかを測る.
    \item[actual results(実行結果)\index{actual results}]test executionの結果として観測される,test itemの振る舞いや状態の集合,関連するデータやtest environmentの状態の集合のこと.例えば,ハードウェアへの出力,データの変更,レポート,メッセージの送信がある.
    \item[backup and recovery testing\index{backup and recovery testing}]reliability testingの一種.時間,コスト,完全性,故障事例の確実性といった特定の状況下でシステムの状態をバックアップから復元できるかどうかを測定するテスト.
    \item[Backus-Naur Form\index{Backus-Naur Form}]formal meta-language used for defining the syntax of a language in a textual format
    \item[base choice\index{base choice}]→base value
    \item[black-box testing(ブラックボックステスト)\index{black-box testing}]→specification-based testing
    \item[c-use\index{c-use}]→computation data use
    \item[capacity testing\index{capacity testing}]performance efficiency testingの一種.ユーザ,トランザクション,データストレージのような負荷の上昇に対し,要求されるパフォーマンスを維持する能力を損なうレベルを測定する.
    \item[compatibility testing(互換性テスト)\index{compatibility testing}]共通の環境(co-existence)や,必要に応じて他のシステムやコンポーネントと情報交換をする状態(interoperability)において,他の独立したプロダクトと共にtest itemが満足に作用するかどうかを測定するテスト.
    \item[completion criteria(終了基準)\index{completion criteria}]テストの処理が完了したと見なされる条件のこと.
    \item[computation data use\index{computation data use}]任意の種類のステートメントにおける,変数の値の使用
    \item[condition(条件)\index{condition}]論理演算子を含まない論理式のこと.{\sf A < B}はconditionだが{\sf A and B}はconditionではない.
    \item[control flow(制御フロー)\index{control flow}]test itemの実行中に処理が実行されるシーケンス.
    \item[control flow sub-path\index{control flow sub-path}]test item中のexectabule statementのシーケンス.
    \item[coverage item(カバレッジアイテム)\index{coverage item}]→test coverage item
    \item[data definition(データ定義)\index{data definition}]値を変数に割り当てるステートメント.別名: variable definition\index{variable definition}.
    \item[data definition c-use pair\index{data definition c-use pair}]data definition及び,後続のcomputation data useでdata useがdata definitionで定義された値を使うもののこと.
    \item[data definition p-use pair\index{data definition p-use pair}]data definition及び,後続のpredicate data useでdata useがdata definitionで定義された値を使うもののこと.
    \item[data definition-use pair\index{data definition-use pair}]data definition及び,後続のdata useでdata useがdata definitionで定義された値を使うもののこと.
    \item[data use\index{data use}]変数の値にアクセスするexecutable statementのこと.
    \item[decision(判定)\index{decision}]処理の集合の出力として考えうる2つ以上の出力のうちどちらかを選ぶ状態のこと.if-then-elseのような単純な選択や,while-loopのようなループをいつ脱出するかの決定,case-1-2-3-..-Nのようなcase(switch)文のこと.
    \item[decision outcome(判定結果)]decisionの結果のこと.これに従ってcontrol flowがどちらに進むかが決まる.
    \item[decision rule\index{decision rule}]decision table testingやcause-effect graphingにおいて,特定の出力を出力するconditionの組み合わせ(= causes)と処理(= effects)のこと.
    \item[definition-use pair(定義使用ペア)\index{definition-use pair}]data definition及び,後続のpredicate/computational data useでdata useがdata definitionで定義された値を使うもののこと.
    \item[domain layer\index{domain layer}]highest level of abstraction for the test item.Note 1 to entry: Keywords on this level are chosen in a way that is familiar to domain experts.
    \item[dynamic testing(動的テスト)\index{dynamic testing}]test itemの実行を必要とするテストのこと.
    \item[endurance testing(耐久テスト)\index{endurance testing}]performance efficiency testingの一種で,test itemが特定の時間の間必要な読み込みを継続的に行えるかどうか確かめるテスト.
    \item[entry point(開始点)\index{entry point}]test itemの実行を開始できるポイントのこと.An entry point is an executable statement within a test item that may be selected by an external process as the starting point for one or more paths through the test item. 大抵はtest item中の最初のexecutable statementである.
    \item[equivalence partition(同値分割)\index{equivalence partition}]同じ区分の全ての値がtest itemによって同様に扱われることが合理的に期待できるような,test itemの範囲内や境界面の変数の変域の部分集合や変数の集合
    \item[equivalence partition coverage(同値分割カバレッジ)\index{equivalence partition coverage}]test setによってカバーされるtest itemの,同一と見なされた同値分割の割合.多くの場合,同値分割の同一確認は(無効な分割の部分分割においては特に)主観的である.よって,test item内の同値分割数の決定的な勘定は不可能である.
    \item[equivalence partitioning(同値分割法)\index{equivalence partitioning}]test design techniqueの1つで,各同値クラス内の1つ以上の代表値を用いることで,equivalence partitionsを実行するテストを用意すること.
    \item[error guessing(エラー推測)\index{error guessing}]テスターの過去の失敗の経験や,失敗の様式の一般的な知識といったものをベースに行うtest design technique.関連する知識は個人的な経験から得られたり,欠陥データベースやバグの分類に要約されたりする.
    \item[executable statement(実行ステートメント)\index{executable statement}]コンパイルされた時にオブジェクトコードに翻訳され,test itemが実行されプログラムデータ上でアクションを起こすだろう時に手続き的に実行されるステートメントのこと.
        statement which, when compiled, is translated into object code, which will be executed procedurally when the test item is running and may perform an action on program data
    \item[exit point(終了点)\index{exit point}]test item中の最後のexecutable statementのこと.An exit point is a terminal point of a path through a test item, being an executable statement within the test item which either terminates the test item, or returns control to an external process. This is most commonly the last executable statement within the test item
    \item[expected results(期待結果)\index{expected results}]仕様や別のソースをもとにした特定の状態の下のtest itemの,観測可能で予想される振る舞い
    \item[exploratory testing(探索的テスト)\index{exploratory testing}]experience-based testingの一種.テスターの関連知識やテストアイテムの事前の探索(前回のテストの結果を含む),一般的なソフトウェアの振る舞いと失敗の種類に対するヒューリスティックな"大まかなやり方"をベースにしてテスターが自発的にテストをデザインし実行するもの.exploratory testingは,温和である可能性がかなり高いと同時に,テスト中ソフトウェアの他のプロパティを邪魔するためにソフトウェアが失敗するリスクを生み出す隠れたプロパティ(隠れた振る舞いを含む)を探す.
    \item[feature set\index{feature set}]リスクや要求,機能,モデルから集めることのできるテスト対象のtest itemのtest conditionを含むアイテムの集合.これは,そのアイテムの全ての機能の集合(full feature set)の場合や,特定の目的のために識別された部分集合(functional feature set)の場合がある.
    \item[high-level keyword\index{high-level keyword}]keyword that covers complex activities that may be composed from other keywords and is used by domain experts to assemble keyword test cases
    \item[Incident Report(インシデントレポート)\index{Incident Report}]インシデントの発生,性質,ステータスのドキュメント
    \item[installability testing(設置性テスト)\index{installability testing}]portability testingの一種.test itemやその集合が全ての指定の環境でインストールできることを確認するテスト
    \item[keyword\index{keyword}]one or more words used as a reference to a specific set of actions intended to be performed during the execution of one or more test cases.Note 1 to entry: The actions include interactions with the User Interface during the test, verification, and specific actions to set up a test scenario.Note 2 to entry: Keywords are named using at least one verb.Note 3 to entry: Composite keywords can be constructed based on other keywords.
    \item[keyword dictionary\index{keyword dictionary}]repository containing a set of keywords reflecting the language and level of abstraction used to write test cases
    \item[Keyword-Driven Testing(キーワード駆動テスト)\index{Keyword-Driven Testing}]testing using test cases composed from keywords
    \item[Keyword-Driven Testing framework\index{Keyword-Driven Testing framework}]test framework covering the functional capabilities of a keyword-driven editor, decomposer, data sequencer, manual test assistant, tool bridge, data and script repositories, a keyword library and the test execution environment
    \item[keyword execution code\index{keyword execution code}]implementation of a keyword that is intended to be executed by a test execution engine
    \item[keyword library\index{keyword library}]→keyword dictionary
    \item[keyword test case\index{keyword test case}]sequence of keywords and the required values for their associated parameters (as applicable) that are composed to describe the actions of a test case
    \item[load testing(ロードテスト)\index{load testing}]type of performance efficiency testing conducted to evaluate the behaviour of a test item under anticipated conditions of varying load, usually between anticipated conditions of low, typical, and peak usage
    \item[low-level keyword\index{low-level keyword}]keyword that covers only one or very few simple actions and is not composed from other keywords
    \item[maintainability testing(保守性テスト)\index{maintainability testing}]test type conducted to evaluate the degree of effectiveness and efficiency with which a test item may be modified
    \item[manual testing\index{manual testing}]人間がtest itemに情報を入力したり結果を検証したりして行うテストのこと.automated testing\index{automated testing}ではツールやロボットや他のtest execution engineを用いてテストする.manual testingではこういったアイテムを使用しない.
    \item[Organizational Test Policy\index{Organizational Test Policy}]an executive-level document that describes the purpose, goals, and overall scope of the testing within an organization, and which expresses why testing is performed and what it is expected to achieve.Note 1 to entry: It is generally preferable to keep the Organizational Test Policy as short as possible in a given context.
    \item[Organizational Test Process\index{Organizational Test Process}]test process for developing and managing organizational test specifications
    \item[organizational test specification\index{organizational test specification}]document that provides information about testing for an organization, i.e. information that is not project-specific. EXAMPLE The most common examples of organizational test specifications are Organizational Test Policy and Organizational Test Strategy.
    \item[Organizational Test Strategy\index{Organizational Test Strategy}]document that expresses the generic requirements for the testing to be performed on all the projects run within the organization, providing detail on how the testing is to be performed.Note 1 to entry: The Organizational Test Strategy is aligned with the Organizational Test Policy.Note 2 to entry: An organisation could have more than one Organizational Test Strategy to cover markedly different project contexts.
    \item[p-use\index{p-use}]→predicate data use
    \item[P-V pair\index{P-V pair}]→test itemのパラメータとそれに割り当てた値の組み合わせのこと.combinational test design technique\index{combinational test design technique}でtest conditionとcoverage itemとして用いる.
    \item[pass/fail criteria(合格/失敗基準)\index{pass/fail criteria}]テスト後にtest itemやその機能が合格か失敗かを決定するのに使われるdecision ruleのこと.
    \item[path(パス)\index{path}]→test itemのexecutable statementsのシーケンスのこと.
    \item[performance testing(性能テスト)\index{performance testing}]type of testing conducted to evaluate the degree to which a test item accomplishes its designated functions within given constraints of time and other resources
    \item[portability testing(移植性テスト)\index{portability testing}]type of testing conducted to evaluate the ease with which a test item can be transferred from one hardware or software environment to another, including the level of modification needed for it to be executed in various types of environment
    \item[predicate(プレディケート)\index{predicate}]{\sf TRUE}または{\sf FALSE}と評価される論理式のことで,通常はコード中の実行パスを指し示す.
    \item[predicate data use\index{predicate data use}]data use associated with the decision outcome of the predicate portion of a decision statement
    \item[procedure testing(手続きテスト)\index{procedure testing}]type of functional suitability testing conducted to evaluate whether procedural instructions for interacting with a test item or using its outputs meet user requirements and support the purpose of their use
    \item[product risk(プロダクトリスク)\index{product risk}]risk that a product may be defective in some specific aspect of its function, quality, or structure
    \item[project risk(プロジェクトリスク)\index{project risk}]risk related to the management of a project.EXAMPLE Lack of staffing, strict deadlines, changing requirements.
    \item[regression testing(回帰テスト)\index{regression testing}]testing following modifications to a test item or to its operational environment, to identify whether regression failures occur.Note 1 to entry: The sufficiency of a set of regression test cases depends on the item under test and on the modifications to that item or its operational environment.
    \item[reliability testing(回復性テスト)\index{reliability testing}]type of testing conducted to evaluate the ability of a test item to perform its required functions, including evaluating the frequency with which failures occur, when it is used under stated conditions for a specified period of time
    \item[retesting\index{retesting}]re-execution of test cases that previously returned a "fail" result, to evaluate the effectiveness of intervening corrective actions.Note 1 to entry: Also known as confirmation testing.
    \item[risk-based testing(リスクベースドテスト)\index{risk-based testing}]testing in which the management, selection, prioritisation, and use of testing activities and resources are consciously based on corresponding types and levels of analyzed risk
    \item[scenario testing(シナリオテスト)\index{scenario testing}]class of test design technique in which tests are designed to execute individual scenarios.Note 1 to entry: A scenario can be a user story, use-case, operational concept, or sequence of events the software may encounter etc.
    \item[scripted testing(スクリプトテスト)\index{scripted testing}]dynamic testingの一種.テスターの操作がtest caseの中の指示書に事前に記述されている.この用語は通常スクリプトで自動で実行されるテストよりも手動で実行されるテストに適用される.
    \item[security testing(セキュリティテスト)\index{security testing}]type of testing conducted to evaluate the degree to which a test item, and associated data and information, are protected so that unauthorized persons or systems cannot use, read, or modify them, and authorized persons or systems are not denied access to them
    \item[specification-based testing(仕様ベースドテスト)\index{specification-based testing}]testing in which the principal test basis is the external inputs and outputs of the test item, commonly based on a specification, rather than its implementation in source code or executable software.Note 1 to entry: Synonyms for specification-based testing include black-box testing and closed box testing.
    \item[statement coverage(ステートメントカバレッジ)\index{statement coverage}]test setによってカバーされた,test itemの全てのexecutable statementの集合の割合のこと.
    \item[statement testing(ステートメントテスト)\index{statement testing}]test design technique in which test cases are constructed to force execution of individual statements in a test item
    \item[static testing(静的テスト)\index{static testing}]コードを実行せずに,品質や他の基準についてtest itemを検査するtestingのこと.レビューや静的解析など.
    \item[stress testing(ストレステスト)\index{stress testing}]type of performance efficiency testing conducted to evaluate a test item's behaviour under conditions of loading above anticipated or specified capacity requirements, or of resource availability below minimum specified requirements
    \item[structural testing(構造テスト)\index{structural testing}]→structure-based testing
    \item[structure-based testing(構造ベースドテスト)\index{structure-based testing}]dynamic testing in which the tests are derived from an examination of the structure of the test item.Note 1 to entry: Structure-based testing is not restricted to use at component level and can be used at all levels, e.g. menu item coverage as part of a system test.Note 2 to entry: Techniques include branch testing, decision testing, and statement testing.Note 3 to entry: Synonyms for structure-based testing are structural testing, glass-box testing, and white box testing.
    \item[sub-path(サブパス)\index{sub-path}]より大きなpathの一部のpathのこと.
    \item[suspension criteria(中止基準)\index{suspension criteria}]criteria used to (temporarily) stop all or a portion of the testing activities
    \item[test basis(テストベース)\index{test basis}]body of knowledge used as the basis for the design of tests and test cases.Note 1 to entry: The test basis may take the form of documentation, such as a requirements specification, design specification, or module specification, but may also be an undocumented understanding of the required behaviour.
    \item[test case(テストケース)]set of test case preconditions, inputs (including actions, where applicable), and expected results, developed to drive the execution of a test item to meet test objectives, including correct implementation, error identification, checking quality, and other valued information.Note 1 to entry: A test case is the lowest level of test input (i.e. test cases are not made up of test cases) for the test sub- process for which it is intended.Note 2 to entry: Test case preconditions include test environment, existing data (e.g. databases), software under test, hardware, etc.Note 3 to entry: Inputs are the data information used to drive test execution.Note 4 to entry: Expected results include success criteria, failures to check for, etc.
    \item[Test Case Specification(テストケース仕様)\index{Test Case Specification}]documentation of a set of one or more test cases
    \item[Test Completion Process\index{Test Completion Process}]Test Management Process for ensuring that useful test assets are made available for later use, test environments are left in a satisfactory condition, and the results of testing are recorded and communicated to relevant stakeholders
    \item[Test Completion Report\index{Test Completion Report}]report that provides a summary of the testing that was performed.Note 1 to entry: Also known as a Test Summary Report.
    \item[test condition(テスト条件)\index{test condition}]testable aspect of a component or system, such as a function, transaction, feature, quality attribute, or structural element identified as a basis for testing.Note 1 to entry: Test conditions can be used to derive coverage items, or can themselves constitute coverage items.
    \item[test coverage(テストカバレッジ)\index{test coverage}]test caseによって実行された特定のtest coverage itemの割合で,パーセントで表される.
    \item[test coverage item\index{test coverage item}]test executionの徹底度合いを測定できるようにするためにtest design techniqueに使われた,1つ以上のtest conditionに由来する属性または属性の集合のこと.
    \item[test data(テストデータ)\index{test data}]1つ以上のtest caseを実行するために入力条件を満たすために作られるか選ばれるデータで,Test Planやtest case,test procedureで定義される.test dataはテスト下のプロダクトとともに(配列やファイル,データベースの形で)保存されるか,他のシステムやそのコンポーネント,ハードウェアデバイス,人間の操作によって与えられる.
    \item[Test Data Readiness Report\index{Test Data Readiness Report}]document describing the status of each test data requirement
    \item[Test Design and Implementation Process\index{Test Design and Implementation Process}]test process for deriving and specifying test cases and test procedures
    \item[Test Design Specification(テスト設計仕様)\index{Test Design Specification}]document specifying the features to be tested and their corresponding test conditions
    \item[test design technique(テスト設計技法)\index{test design technique}]activities, concepts, processes, and patterns used to construct a test model that is used to identify test conditions for a test item, derive corresponding test coverage items, and subsequently derive or select test cases
    \item[test environment(テスト環境)\index{test environment}]facilities, hardware, software, firmware, procedures, and documentation intended for or used to perform testing of software.Note 1 to entry: A test environment could contain multiple environments to accommodate specific test sub-processes (e.g. a unit test environment, a performance test environment etc.).
    \item[test environment readiness report\index{test environment readiness report}]document that describes the fulfilment of each test environment requirement
    \item[Test Environment Requirements\index{Test Environment Requirements}]description of the necessary properties of the test environment.Note 1 to entry: All or parts of the test environment requirements could reference where the information can be found, e.g. in the appropriate Organizational Test Strategy, Test Plan, and/or Test Specification.
    \item[Test Environment Set-up Process\index{Test Environment Set-up Process}]dynamic test process for establishing and maintaining a required test environment
    \item[test execution(テスト実行)\index{test execution}]test itemに対してテストを実行してactual resultを得ること.
    \item[test execution engine\index{test execution engine}]test caseを実行するためにtest itemを操作する,ソフトウェアやハードウェアに実装されているツール.典型的なtest execution engineはunit test tool framework,stimulation-command system,capture and playback toolや操作のためのソフトウェアが付属したhardware robotを含む.
    \item[Test Execution Log\index{Test Execution Log}]document that records details of the execution of one or more test procedures
    \item[Test Execution Process\index{Test Execution Process}]dynamic test process for executing test procedures created in the Test Design and Implementation Process in the prepared test environment, and recording the results
    \item[test framework\index{test framework}]テストを容易にする環境のこと.
    \item[Test Incident Reporting Process\index{Test Incident Reporting Process}]dynamic test process for reporting to the relevant stakeholders issues requiring further action that were identified during the test execution process
    \item[test interface\index{test interface}]interface to the test item used to stimulate the test item, to get responses (e.g. actual results), or both.Note 1 to entry: The GUI, API or SOA interfaces are typical test interfaces.Note 2 to entry: Stimulating the test item can involve passing data into it via computer interfaces or attached hardware. Note 3 to entry: Getting responses includes getting information from the test item under test or associated hardware.
    \item[test interface layer\index{test interface layer}]lowest level of abstraction for keywords, which interacts with the test item directly and encapsulates the atomic (lowest level) interactions at the test interface
    \item[test item(テストアイテム)\index{test item}]テスト対象の作業成果物.システムやソフトウェアアイテム,要求仕様書,設計仕様書,ユーザガイドなど.
    \item[test level(テストレベル)\index{test level}] test sub-processの具体的なインスタンス化.次にあげたものはtest sub-processとしてインスタンス化される一般的なtest levelである: component test level/sub-process,integration test level/sub-process,system test level/sub-process,acceptance test level/sub-process.test levelはtest phaseの代名詞である.
    \item[test management(テストマネジメント)\index{test management}]planning, scheduling, estimating, monitoring, reporting, control and completion of test activities
    \item[Test Management Process\index{Test Management Process}]test process containing the sub-processes that are required for the management of a test project.Note 1 to entry: See Test Planning Process, Test Monitoring and Control Process, Test Completion Process.
    \item[test model\index{test model}]representation of a test item that is used during the test case design process
    \item[Test Monitoring and Control Process\index{Test Monitoring and Control Process}]Test Management Process for ensuring that testing is performed in line with a Test Plan and with organizational test specifications
    \item[test object(テスト対象)\index{test object}]→test item
    \item[test phase(テストフェーズ)\index{test phase}]specific instantiation of test sub-process.Note 1 to entry: Test phases are synonymous with test levels, therefore examples of test phases are the same as for test levels (e.g. system test phase/sub-process).
    \item[Test Plan(テスト計画)\index{Test Plan}]detailed description of test objectives to be achieved and the means and schedule for achieving them, organised to coordinate testing activities for some test item or set of test items.Note 1 to entry: A project can have more than one Test Plan, for example there could be a Project Test Plan (also known as a master test plan) that encompasses all testing activities on the project; further detail of particular test activities could be defined in one or more test sub-process plans (i.e. a system test plan or a performance test plan).Note 2 to entry: Typically a Test Plan is a written document, though other plan formats could be possible as defined locally within an organization or project.Note 3 to entry: Test Plans could also be written for non-project activities, for example a maintenance test plan.
    \item[Test Planning Process\index{Test Planning Process}]Test Management Process used to complete test planning and develop Test Plans
    \item[Test Policy(テストポリシー)\index{Test Policy}]an executive-level document that describes the purpose, goals, principles and scope of testing within an organization.Note 1 to entry: The Test Policy defines what testing is performed and what it is expected to achieve but does not detail how testing is to be performed.Note 2 to entry: The Test Policy can provide a framework for establishing, reviewing and continually improving the organisations testing.
    \item[test practice\index{test practice}]conceptual framework that can be applied to the Organizational Test Process, the Test Management Process, and/or the Dynamic Test Process to facilitate testing.Note 1 to entry: Test Practices are sometimes referred to as test approaches.
    \item[test procedure(テスト手順)\index{test procedure}]実行順序におけるtest caseのシーケンスと,最初の初期条件のセットアップで必要になる関連する処理,実行後の処理をまとめるもののこと.test procedureは,連続して実行するために選んだ1つ以上のtest caseの集合の実行方法の詳細な指示を含む.sequence of test cases in execution order, and any associated actions that may be required to set up the initial preconditions and any wrap up activities post execution.Note 1 to entry: Test procedures include detailed instructions for how to run a set of one or more test cases selected to be run consecutively, including set up of common preconditions, and providing input and evaluating the actual result for each included test case.
    \item[Test Procedure Specification(テスト手順仕様)\index{Test Procedure Specification}]document specifying one or more test procedures, which are collections of test cases to be executed for a particular objective.Note 1 to entry: The test cases in a test set are listed in their required order in the test procedure..Note 2 to entry: Also known as a manual test script. A test procedure specification for an automated test run is usually called a test script.
    \item[test process(テストプロセス)\index{test process}]しばしば多くのアクティビティからなり,1つ以上のtest sub-processに分割される,ソフトウェアプロダクトの品質に関する情報を提供する.provides information on the quality of a software product, often comprised of a number of activities, grouped into one or more test sub-processes.特定のプロジェクトのtest processはシステムtest sub-processやtest planning sub-process(より大きなtest management processの一部),static testing sub-processといった複数のサブプロセスから構成されることもありうる.
    \item[test requirement(テスト要件)\index{test requirement}]→test condition
    \item[test result(テスト結果)\index{test result}]indication of whether or not a specific test case has passed or failed, i.e. if the actual result observed as test item output corresponds to the expected result or if deviations were observed
    \item[test script(テストスクリプト)\index{test script}]manual/automated testingのためのtest procedure specification.
    \item[test set(テストセット)\index{test set}]1つまたはそれ以上のtest caseとその実行に対する一般的な制約の集合のこと.特定のtest environment.特殊なドメイン知識,特定の目的など.
    \item[test specification(テスト仕様書)\index{test specification}]complete documentation of the test design, test cases and test procedures for a specific test item.Note 1 to entry: A test specification could be detailed in one document, in a set of documents, or in other ways, for example in a mixture of documents and database entries.
    \item[test specification technique(テスト仕様化技法)\index{test specification technique}]→test design technique
    \item[test status report\index{test status report}]report that provides information about the status of the testing that is being performed in a specified reporting period
    \item[test strategy(テスト戦略)\index{test strategy}]part of the Test Plan that describes the approach to testing for a specific test project or test sub-process or sub-processes.Note 1 to entry: The test strategy is a distinct entity from the Organizational Test Strategy..Note 2 to entry: The test strategy usually describes some or all of the following: the test practices used; the test sub- processes to be implemented; the retesting and regression testing to be employed; the test design techniques and corresponding test completion criteria to be used; test data; test environment and testing tool requirements; and expectations for test deliverables.
    \item[test sub-process\index{test sub-process}]一般的にtest projectの全test process内にある,特定のtest level(システムテスト,acceptance testingなど)やtest type(usability testing,performance testingなど)を行うために使うtest managementとdynamic/static test processのこと.test sub-processは1つまたはそれ以上のtest typeを含む.ライフサイクルモデルに頼ると,test sub-processは一般的にtest phase,test level,test stage,test taskと呼ばれる.
    \item[test technique(テスト技法)\index{test technique}]→test design technique
    \item[test traceability matrix\index{test traceability matrix}]document, spreadsheet, or other automated tool used to identify related items in documentation and software, such as requirements with associated tests.別名verification cross reference matrix\index{verification cross reference matrix},requirements test matrix\index{requirements test matrix},requirements verification table\index{requirements verification table}など.異なるtest traceability matrixは異なる情報,フォーマット,詳細のレベルを含む.
    \item[test type(テストタイプ)\index{test type}]特定の品質特性に着目したtesting activitiesの集合.test typeは1つのtest sub-processで行われるか,たくさんのtest sub-process(performance testingがcomponent test sub-processやsystem test sub-processとともに行われるなど)を横断して行われる.たとえば,security testing,functional testing,useability testing,performance testingなど.
    \item[testing(テスト)\index{testing}]set of activities conducted to facilitate discovery and/or evaluation of properties of one or more test items.Note 1 to entry: Testing activities could include planning, preparation, execution, reporting, and management activities, insofar as they are directed towards testing.
    \item[testware(テストウェア)\index{testware}]artefacts produced during the test process required to plan, design, and execute tests.Note 1 to entry: Testware can include such things as documentation, scripts, inputs, expected results, files, databases, environment, and any additional software or utilities used in the course of testing.
    \item[unscripted testing\index{unscripted testing}]dynamic testingの一種で,事前にtest case中にテスターの行動を記さないもののこと.
    \item[variable definition\index{variable definition}]→data definition
    \item[volume testing(ボリュームテスト)\index{volume testing}]volume testing type of performance efficiency testing conducted to evaluate the capability of the test item to process specified volumes of data (usually at or near maximum specified capacity) in terms of throughput capacity, storage capacity, or both
    \item[white box testing(ホワイトボックステスト)\index{white box testing}]→structure-based testing
\end{description}


